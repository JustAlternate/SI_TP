\documentclass{article}

 \usepackage[margin=0.5in]{geometry}
\usepackage{amsmath}
\usepackage{listings}

\author{Loïc Weber \& Thibault Gounant} 
\title{Cahier de TP}

\begin{document}

\setlength\parindent{0pt}

\maketitle

\renewcommand*\contentsname{Table des matières}
\tableofcontents

\section{TP1 : Unix}

\subsection{Déplacement en UNIX :}

\begin{lstlisting}[language=bash]

/    # repertoire racine
pwd  # repertoire courant
..   # repertoire parent
~    # repertoire maison
cd   # change le repertoire

\end{lstlisting}

\subsection{Commandes fréquentes :}

\begin{lstlisting}[language=bash]

ls                              # affiche les fichiers du repertoire courant
ls -l                           # affiche les fichiers avec plus de details
mkdir dossier                   # creer un dossier
touch fichier                   # creer un fichier vide
cp fichier1 dossier/fichier2    # copie le fichier1 dans un dossier sous le nom fichier2
mv fichier1 dossier/            # deplace le fichier1 dans le dossier
cat fichier                     # affiche le contenu du fichier
rm fichier1 fichier2            # supprime les fichiers 1 et 2
rm dossier -R                   # supprime un dossier recursivement

\end{lstlisting}

\subsection{Autres commandes :}

\subsubsection{apropos}
Affiche les noms et descriptions des manuels en rapport.

\begin{lstlisting}[language=bash]

apropos <mot> <options>
apropos mkdir  # mkdir(1) ...., mkdir(2) ...., mkdirat(2) ....
apropos printf # printf(1) ...., sprintf(1)..., eprintf(1) ....

\end{lstlisting}

\subsubsection{whatis}
Affiche la description des de la page du manuel de la commande cherché

\begin{lstlisting}[language=bash]
whatis <mot> <options>
whatis cd   # donne la description de la commande cd.
\end{lstlisting}

\subsubsection{grep}
Permet la recherche de motifs dans un fichier ou dans plusieurs fichiers.

\begin{lstlisting}[language=bash]

grep bonjour fichier.txt                # affiche toute les lignes avec le mot bonjour
grep bonjour fichier.txt --count        # compte le nombre de lignes avec le mot bonjour
grep bonjour fichier.txt --invert-match # les lignes qui ne contiennent pas le mot bonjour

\end{lstlisting}

\subsubsection{file}
Permet d'avoir une description du type d'un fichier donné en argument.


\begin{lstlisting}[language=bash]

file latex.tx  # latex.tex: LaTeX 2e document, Unicode text, UTF-8 text

\end{lstlisting}



\end{document}
