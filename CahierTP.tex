\documentclass{article}

 \usepackage[margin=0.5in]{geometry}
\usepackage{amsmath}
\usepackage{listings}
\usepackage[utf8]{inputenc}
\usepackage[T1]{fontenc}


\author{Loïc Weber \& Thibault Gounant} 
\title{Cahier de TP}

\begin{document}

\setlength\parindent{0pt}

\maketitle

\renewcommand*\contentsname{Table des matières}
\tableofcontents

\section{TP1 : Unix}

\subsection{Manuel}
\begin{lstlisting}[language=bash]
man <commande>              # Afficher les pages du manuel correspondant à la commande
whatis <commande>           # Afficher la description du manuel correspondant à la commande
apropos <mot>               # Afficher les descriptions du manuel dont les pages contiennent le mot 
history                     # Afficher l'historique des commandes
alias <commande>=<expr>     # Créer une commande correspondant à une expression
\end{lstlisting}

\subsection{Hierarchie}
\begin{lstlisting}[language=bash]
/    # repertoire racine
..   # repertoire parent
~    # repertoire maison
\end{lstlisting}

\begin{lstlisting}[language=bash]
ls <dossier>    # Afficher le contenu du dossier
pwd             # Afficher le chemin absolu du dossier courant
cd <dossier>    # Changer de dossier
\end{lstlisting}

\subsection{Affichage}
\begin{lstlisting}[language=bash]
file <fichier>  # Afficher une description du type du fichier
cat <fichier>   # Afficher le contenu du fichier
sort <fichier>  # Afficher le contenu du fichier trié
more <fichier>  # Visualiser le contenu du fichier par le haut
less <fichier>  # Visualiser le contenu du fichier par le bas
\end{lstlisting}

\subsection{Gestion}
\begin{lstlisting}[language=bash]
chmod <droit> <fichier>     # Changer les droits du fichier
mkdir <dossier>             # Créer le dossier
rmdir <dossier>             # Supprimer le dossier vide
touch <fichier>             # Créer le fichier
rm <fichier>                # Supprimer le fichier
cp <fichier> <dossier>      # Copier le fichier dans le dossier
mv <fichier> <dossier>      # Déplacer le fichier dans le dossier
ln <fichier1> <fichier2>    # Créer un raccourci fichier1 vers le fichier2 
\end{lstlisting}

\subsection{Recherche}
\begin{lstlisting}[language=bash]
find <expr>             # Rechercher un fichier
grep <expr> <fichier>   # Rechercher une expression dans le fichier
\end{lstlisting}

\end{document}
